%% This template can be used to write a paper for
%% Computer Physics Communications using LaTeX.
%% For authors who want to write a computer program description,
%% an example Program Summary is included that only has to be
%% completed and which will give the correct layout in the
%% preprint and the journal.
%% The `elsarticle' style is used and more information on this style
%% can be found at 
%% http://www.elsevier.com/wps/find/authorsview.authors/elsarticle.
%%
%%
\documentclass[preprint,12pt]{elsarticle}

%% Use the option review to obtain double line spacing
%% \documentclass[preprint,review,12pt]{elsarticle}

%% Use the options 1p,twocolumn; 3p; 3p,twocolumn; 5p; or 5p,twocolumn
%% for a journal layout:
%% \documentclass[final,1p,times]{elsarticle}
%% \documentclass[final,1p,times,twocolumn]{elsarticle}
%% \documentclass[final,3p,times]{elsarticle}
%% \documentclass[final,3p,times,twocolumn]{elsarticle}
%% \documentclass[final,5p,times]{elsarticle}
%% \documentclass[final,5p,times,twocolumn]{elsarticle}

%% if you use PostScript figures in your article
%% use the graphics package for simple commands
%% \usepackage{graphics}
%% or use the graphicx package for more complicated commands
%% \usepackage{graphicx}
%% or use the epsfig package if you prefer to use the old commands
%% \usepackage{epsfig}

%% The amssymb package provides various useful mathematical symbols
\usepackage{amssymb}
%% The amsthm package provides extended theorem environments
%% \usepackage{amsthm}

%% The lineno packages adds line numbers. Start line numbering with
%% \begin{linenumbers}, end it with \end{linenumbers}. Or switch it on
%% for the whole article with \linenumbers after \end{frontmatter}.
%% \usepackage{lineno}

%% natbib.sty is loaded by default. However, natbib options can be
%% provided with \biboptions{...} command. Following options are
%% valid:

%%   round  -  round parentheses are used (default)
%%   square -  square brackets are used   [option]
%%   curly  -  curly braces are used      {option}
%%   angle  -  angle brackets are used    <option>
%%   semicolon  -  multiple citations separated by semi-colon
%%   colon  - same as semicolon, an earlier confusion
%%   comma  -  separated by comma
%%   numbers-  selects numerical citations
%%   super  -  numerical citations as superscripts
%%   sort   -  sorts multiple citations according to order in ref. list
%%   sort&compress   -  like sort, but also compresses numerical citations
%%   compress - compresses without sorting
%%
%% \biboptions{comma,round}

% \biboptions{}

%% This list environment is used for the references in the
%% Program Summary
%%
\newcounter{bla}
\newenvironment{refnummer}{%
\list{[\arabic{bla}]}%
{\usecounter{bla}%
 \setlength{\itemindent}{0pt}%
 \setlength{\topsep}{0pt}%
 \setlength{\itemsep}{0pt}%
 \setlength{\labelsep}{2pt}%
 \setlength{\listparindent}{0pt}%
 \settowidth{\labelwidth}{[9]}%
 \setlength{\leftmargin}{\labelwidth}%
 \addtolength{\leftmargin}{\labelsep}%
 \setlength{\rightmargin}{0pt}}}
 {\endlist}

\journal{Computer Physics Communications}

\begin{document}

\begin{frontmatter}

%% Title, authors and addresses

%% use the tnoteref command within \title for footnotes;
%% use the tnotetext command for the associated footnote;
%% use the fnref command within \author or \address for footnotes;
%% use the fntext command for the associated footnote;
%% use the corref command within \author for corresponding author footnotes;
%% use the cortext command for the associated footnote;
%% use the ead command for the email address,
%% and the form \ead[url] for the home page:
%%
%% \title{Title\tnoteref{label1}}
%% \tnotetext[label1]{}
%% \author{Name\corref{cor1}\fnref{label2}}
%% \ead{email address}
%% \ead[url]{home page}
%% \fntext[label2]{}
%% \cortext[cor1]{}
%% \address{Address\fnref{label3}}
%% \fntext[label3]{}

% \title{WaterLily.jl: A differentiable fluid simulator in Julia with fast heterogeneous execution}
\title{WaterLily.jl: Simulating fluid flow and dynamic geometries with a backend-agnostic Julia solver}

%% use optional labels to link authors explicitly to addresses:
%% \author[label1,label2]{<author name>}
%% \address[label1]{<address>}
%% \address[label2]{<address>}

\author[a]{First Author\corref{author}}
\author[a,b]{Second Author}
\author[b]{Third Author}

\cortext[author] {Corresponding author.\\\textit{E-mail address:} firstAuthor@somewhere.edu}
\address[a]{First Address}
\address[b]{Second Address}

% A submitted program is expected to satisfy the following criteria: it must be of benefit to other physicists, or be an exemplar of good programming practice, or illustrate new or novel programming techniques which are of importance to computational physics community; it should be implemented in a language and executable on hardware that is widely available and well documented; it should meet accepted standards for scientific programming; it should be adequately documented and, where appropriate, supplied with a separate User Manual, which together with the manuscript should make clear the structure, functionality, installation, and operation of the program.

% Your manuscript and figure sources should be submitted through Editorial Manager (EM) by using the online submission tool at \\
% https://www.editorialmanager.com/comphy/.

% In addition to the manuscript you must supply: the program source code; a README file giving the names and a brief description of the files/directory structure that make up the package and clear instructions on the installation and execution of the program; sample input and output data for at least one comprehensive test run; and, where appropriate, a user manual.

% A compressed archive program file or files, containing these items, should be uploaded at the "Attach Files" stage of the EM submission.

% For files larger than 1Gb, if difficulties are encountered during upload the author should contact the Technical Editor at cpc.mendeley@gmail.com.

\begin{abstract}
Integrating computational fluid dynamics (CFD) software into optimization and machine-learning frameworks is hampered by the rigidity of classic computational languages and the slow performance of more flexible high-level languages. WaterLily.jl is an open-source incompressible viscous flow solver written in the Julia language. The small code base is multi-dimensional, multi-platform and backend-agnostic (serial CPU, multi-threaded, \& GPU execution). The computational time per time step scales linearly with the number of degrees of freedom on CPUs, and we see up to a 182x speed-up using CUDA kernels. This leads to comparable performance with Fortran solvers on many research-scale problems opening up exciting possible future applications on the cutting edge of machine-learning research.
% The simulator is differentiable and uses automatic-differentiation internally to immerse solid geometries and optimize the pressure solver.
\end{abstract}

\begin{keyword}
heterogeneous-programming; Cartesian-grid methods; Julia; GPU
\end{keyword}

\end{frontmatter}

%%
%% Start line numbering here if you want
%%
% \linenumbers

% All CPiP articles must contain the following
% PROGRAM SUMMARY.

% {\bf PROGRAM SUMMARY/NEW VERSION PROGRAM SUMMARY}
  %Delete as appropriate.
{\bf PROGRAM SUMMARY}
\\
\\
\begin{small}
\noindent
{\em Program Title:} WaterLily.jl \\ 
{\em CPC Library link to program files:} (to be added by Technical Editor) \\
{\em Developer's repository link:} https://github.com/weymouth/WaterLily.jl \\
{\em Code Ocean capsule:} (to be added by Technical Editor)\\
{\em Licensing provisions(please choose one):} MIT \\
{\em Programming language:} Julia \\
{\em Supplementary material:} \\
  % Fill in if necessary, otherwise leave out.
% {\em Journal reference of previous version:}*                  \\
  %Only required for a New Version summary, otherwise leave out.
% {\em Does the new version supersede the previous version?:}*   \\
  %Only required for a New Version summary, otherwise leave out.
% {\em Reasons for the new version:*}\\
  %Only required for a New Version summary, otherwise leave out.
% {\em Summary of revisions:}*\\
  %Only required for a New Version summary, otherwise leave out.
{\em Nature of problem(approx. 50-250 words):}\\
  %Describe the nature of the problem here. \\
{\em Solution method(approx. 50-250 words):}\\
  %Describe the method solution here.
{\em Additional comments including restrictions and unusual features (approx. 50-250 words):}\\
  %Provide any additional comments here.

\begin{thebibliography}{0}
\bibitem{1}Program summary reference 1         % This list should only contain those items referenced in the                 
\bibitem{2}Program summary reference 2         % Program Summary section.   
\bibitem{3}Program summary reference 3         % Type references in text as [1], [2], etc.
                               % This list is different from the bibliography at the end of 
                               % the Long Write-Up.
\end{thebibliography}
% * Items marked with an asterisk are only required for new versions
% of programs previously published in the CPC Program Library.\\
\end{small}


%% main text
\section{Introduction}

Similar to the arXiv paper, and mention some other codes here too. Some references published at CPC that we could cite and follow to organise this paper:

\begin{itemize}
  \item \cite{Romero2022}
  \item \cite{Bernardini2021,Bernardini2023}
  \item \cite{Gasparino2023}
\end{itemize}

\section{Numerical methods}
\begin{itemize}
  \item FV, BDIM, GMG
\end{itemize}

\section{Software design}
\begin{itemize}
  \item N-dimensional solver
  \item KernelAbstractions.jl
  \item Auto bodies
\end{itemize}

\section{Benchmark and validation}
\begin{itemize}
  \item Cases: TGV (no body) [benchmark and validation], sphere/cylinder [benchmark and validation], moving body (?) [benchmark]
  \item Architectures: NVIDIA H100 (Marenostrum 5 at BSC), AMD Radeon Instinct MI50 (CTE-AMD cluster at BSC), serial execution (?)
\end{itemize}

Try different grid sizes, testing the GPUs at different capacity.

Profiling with Nsight (see \cite{Gasparino2023}) for kernel time, CPU-GPU communication.

\section{Flashy test cases}
For example...
\begin{itemize}
  \item 2D pitching/heaving airfoils (parametric bodies)
  \item Jellyfish (expanding/contracting bodies)
  \item RL control (Arthur case) to show integration of CFD and ML?
\end{itemize}

\section{Conclusions}


%% The Appendices part is started with the command \appendix;
%% appendix sections are then done as normal sections
%% \appendix

%% \section{}
%% \label{}

%% References
%%
%% Following citation commands can be used in the body text:
%% Usage of \cite is as follows:
%%   \cite{key}         ==>>  [#]
%%   \cite[chap. 2]{key} ==>> [#, chap. 2]
%%

%% References with bibTeX database:

\bibliographystyle{elsarticle-num}
\bibliography{main.bib}

\end{document}

%%
%% End of file 